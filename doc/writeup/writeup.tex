\documentclass[11pt]{article}
%\input epsf
\usepackage{hyperref}
\usepackage{graphicx}
\usepackage{amsmath}
\usepackage{amssymb}
\usepackage{multicol}
\usepackage{color}
\usepackage{subfigure}
\oddsidemargin=-.125in
\evensidemargin=-.125in
\textwidth=6.5in
\topmargin=-0.5in
\textheight=8.5in
\parskip 3pt
\nonfrenchspacing

\newcommand{\grad}{\vec{\nabla}}
\newcommand{\ibold}{{\boldsymbol{i}}}
\newcommand{\jbold}{{\boldsymbol{j}}}
\newcommand{\kbold}{{\boldsymbol{k}}}
\newcommand{\lbold}{{\boldsymbol{l}}}
\newcommand{\Gbold}{{\boldsymbol{G}}}
\newcommand{\nbold}{{\boldsymbol{n}}}
\newcommand{\ebold}{{\boldsymbol{e}}}
\newcommand{\ed}{{\boldsymbol{e}^d}}
\newcommand{\Zbold}{{\boldsymbol{0}}}
\newcommand{\ubold}{{\boldsymbol{u}}}
\newcommand{\xbold}{{\boldsymbol{x}}}
\newcommand{\ybold}{{\boldsymbol{y}}}
\newcommand{\pbold}{{\boldsymbol{p}}}
\newcommand{\qbold}{{\boldsymbol{q}}}
\newcommand{\rbold}{{\boldsymbol{r}}}
\newcommand{\unitV}{\mathds{1}}
\newcommand{\eb}{\text{EB}}
\newcommand{\vol}{\mathcal{V}}
\newcommand{\face}{\mathcal{F}}
\newcommand{\zerobold}{{\boldsymbol{0}}}
\newcommand{\xbar}{{\bar{\boldsymbol{x}}}}
\newcommand{\half}{{\frac{1}{2}}}
%\newcommand{\Dim}{{\mathbf{D}}}
\newcommand{\Dim}{D}
\newcommand{\dif}{\mathrm{d}}
\newcommand{\dt}{{\Delta t}}
% \newcommand{\avg}[1]{\overline{#1}}
\newcommand{\avg}[1]{\left\langle #1 \right\rangle}
\newcommand{\avgI}[1]{\langle #1 \rangle_{\ibold}}
% i plus, minus, +- half e^d
\newcommand{\iphed}{{\ibold+\frac{1}{2}\ebold^d}}
\newcommand{\imhed}{{\ibold-\frac{1}{2}\ebold^d}}
\newcommand{\ipmhed}{{\ibold\pm\frac{1}{2}\ebold^d}}

% i plus fractional e^d
\newcommand{\ipfed}[1]{\ibold+\frac{#1}{2}\ebold^d}
% i minus fractional e^d
\newcommand{\imfed}[1]{\ibold-\frac{#1}{2}\ebold^d}
\newcommand{\ipmfed}[1]{\ibold\pm\frac{#1}{2}\ebold^d}
\newcommand{\eq}[1]{(\ref{#1})}

\newcommand{\nref}{n_{\text{ref}}}
\newcommand{\phio}{\phi^{\circ}}



\begin{document}

\section*{AMR Subcycled Miniapp Algorithm Description \\
          Bryce Lelbach and Hans Johansen}

\section{Mini-app: Higher-order ImEx Advection-Diffusion Equation}

\subsection{Domain}
Define the 3D domain $\Omega$ with the following assumptions:
\begin{itemize}
  \item Rectangular domain decomposed into $N_l$ levels, each
    consisting of a disjoint set of boxes, and $l = 0$ base level covering
    the entire domain.
  \item Each box is discretized into cells size $h$, indexed
    on that level with $(i,j,k)$ in the $(x,y,z)$ directions.
  \item Each box has $N_g$ ghost cells (including corners) to communicate, 
    which must match the stencil size of the operators, and have proper
    nesting between levels 
    (ghost cells must be filled by stencils on one level).
\end{itemize}

\subsection{Equations and finite volume discretization}
Flux-divergence form of a advection-diffusion equation would
  have . :
\begin{equation}
\label{eqn:adv-diff}
  \partial_t \phi = \nabla \cdot \left( - \vec{u} \phi \right) 
    + \nu \Delta \phi \, .
\end{equation}
We can split the spatial operators in \eqref{eqn:adv-diff} into 
  horizontal $(x,y)$ and vertical $(z)$ operators that contain
  only some of the terms:
\begin{align}
\label{eqn:advxy-diffz}
  \partial_t \phi 
    & = \partial_x \left( - u_x \phi \right) 
      + \partial_y \left( - u_y \phi \right) 
      + \partial_z \left( \nu \partial_z \phi \right)  \, .
\end{align}
For a finite volume method, we integrate this over a cell $V_i$ and
  from $t^n$ to $t^{n+1}$ to obtain an
  equation for the evolution of the cell averages in terms of fluxes:
\begin{align}
   \avg{\phi^{n+1}} - \avg{\phi^n} 
    & = \dt \sum_d \frac{1}{h_d} 
    \left( \avg{F}_\iphed - \avg{F}_\imhed \right) \, ,
\end{align}
  where horizontal fluxes $F_x = - u_x \phi$ and $F_y = - u_y \phi$, and 
  vertical $F_z = \nu \partial_{z} \phi$.
Note that quantities on the right hand side are averaged
  both in space and in time (from ${n \rightarrow {n+1}}$).

\subsection{Calculating fluxes}

\subsubsection{Horizontal fluxes}
Assuming a constant horizontal velocity $\vec{u} = (u_x, u_y)$,
  with $u_x, u_y \ge 0$,
  a fifth-order accurate upwind flux stencil is:
\begin{align}
  \nonumber
  \avg{F_x}_{i - \half} & = \avg{-u_x \phi}_{i - \half} \\
\label{eqn:oh5upwind}
  & = (-u_x) 
  \left( 
    +2 \phi_{i - 3}
    -13 \phi_{i - 2}
    +47 \phi_{i - 1}
    +27 \phi_{i}
    -3 \phi_{i + 1}
  \right)
  \frac{1}{60}
  + O(h^5) \, .
\end{align}
The stencil for $F_y$ is similar. Note that these formula require
  $N_g = 3$ ghost cells, in each direction, but no corner cells.

\subsection{Vertical fluxes and solve}
In stencil form, if $\nu$ is a constant, the vertical operator is:
\begin{align}
  \avg{F_z}_{k - \half} & = \avg{\nu \, \partial_z \phi}_{k - \half} \\ 
   & = \frac{\nu}{h} \left( \phi_{k} - \phi_{k-1} \right) + O(h^2) \, ,
\end{align}
  and we assume $F_z = 0$ at the top and bottom boundaries.
We wish to advance this part implicitly, so the time integration 
  requires we solve the following system at each time step:
\begin{align}
 \phi^{n+1}_k & = 
    \phi^n_k + \frac{\dt}{h} 
    \left(F^{n+1}_{k+\half} - F^{n+1}_{k-\half} \right) 
\\
 & = \phi^n_k +
    \frac{\nu \dt}{h^2} 
    \left(\phi^{n+1}_{k+1}-2\phi^{n+1}_{k}+\phi^{n+1}_{k-1}\right) \, .
\end{align}
This leads to a simple tri-diagonal system to be solved
  for each time step on each $(i,j)$ column of $\phi^{n+1}_k$ values:
\begin{align}
\label{eqn:vertsolve}
  \begin{bmatrix}
    (1 + \alpha) & -\alpha & 0 & 0 & \dots  \\
    -\alpha & (1 + 2 \alpha) & -\alpha & 0 & \dots  \\
    \dots & -\alpha & (1 + 2 \alpha) & -\alpha & \dots  \\
    \dots & 0 & -\alpha & (1 + 2 \alpha) & -\alpha \\
    \dots & 0 &  0 & -\alpha & (1 + \alpha) \\
  \end{bmatrix}
  \phi^{n+1}_k = \phi^n_k
\end{align}
  where $\alpha = \frac{\nu \dt}{h^2}$.

\subsection{Analytic solution}
In a domain $x,y,z \in [0,1]$, which is periodic only in the $x,y$ 
  directions, we can assume a solution of the form:
\begin{align}
\nonumber
  \phi(x,y,z,t) & = V(z,t) + H(x,y,t) \\
\label{eqn:exact}
  & = e^{-\omega_z t} \cos(\pi c_z z)
    + \sin(2\pi (c_x x + c_y y) - \omega_{xy} t) \, ,
\end{align}
  where $\omega_z = \nu \pi^2 c_z^2$ and 
  $\omega_{xy} = u_x c_x + u_y c_y$, where $c \in \{0, 1, 2, \dots\}$.
Subsituting this into \eqref{eqn:advxy-diffz} shows that it is an exact
  solution of our model equation.

\subsubsection{Tests}
For the vertical operator, a simple test is to do just the 
  implicit solve starting with $\phi^n = V(z,0)$, evaluated
  at cell centers (not sure about cell average initialization).
Discrete Fourier analysis on \eqref{eqn:vertsolve} concludes that
  the vertical operator acting on $V(z,0)$ is:
\begin{gather}
  L_{z} V(x,0) =  \gamma_z \cos(\pi c_z z)
  \, , \hbox{ where} \\
  \gamma_z = \frac{\nu}{h^2} (2 \cos(\pi c_z h) - 2) \, .
\end{gather}
Doing the implicit solve \eqref{eqn:vertsolve}
  on this solution $\phi^n = V(z,0)$
  implies $\phi^{n+1} = \left(1 - \dt \gamma_z \right)^{-1} \phi^n $,
  to round-off.

Similarly, for $\phi^n = H(x,y,0)$, the advection operator
  using the fifth-order stencil \eqref{eqn:oh5upwind} can be written as
  an expression that depends on $(u_x,u_y)$, and the ``shift'' $s$
  of the horizontal operator stencil:
\begin{equation}
  L_{xy} H(x,y,0) = 
    -u_x \sum_s q_s \sin(2\pi c_x (x + sh) + 2\pi c_y y) \, 
    -u_y \sum_s q_s \sin(2\pi c_x x + 2\pi c_y (y + sh)) 
  \, ,
\end{equation}
  where for $s \in \{-3, \dots, 2\}, q_s = (-2, 15, -60, 20, 30, -3)/(60 h)$
  is the difference of the face flux stencil, \eqref{eqn:oh5upwind}.
This should be accurate to round-off.

\subsection{Time integration}
The horizontal operator requires RK3, RK4 or ARK4, due to its minimal 
  dissipation and dispersion properties.
A CFL of $\frac{\dt |\vec{u}|}{h} \le \half$ should be stable.

\section{Variations}

\subsection{Horizontal fluxes - spatially-varying velocity}
This is what the dycore does, which requires corner ghost cells
\begin{equation}
  \avg{- u_d \, \partial_d \phi}_\iphed = \avg{u_d}_\iphed 
    \, \avg{\partial_d \phi}_\iphed + 
  \frac{1}{12}\sum_{d'\neq d}
  h_{d'}^2 \, {\partial_{d'} u_d} \, {\partial_{d' d} \phi}  +
  O(h^4) \\
\end{equation}


\end{document}
